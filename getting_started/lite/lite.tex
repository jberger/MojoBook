\section{Your First Lite App}

%XXX consider making an architecture section

When designing a Mojolicious application, authors usually follow the accepted Model-View-Controller architecture.
%TODO external links to MVC
The model layer is intentionally unspecified.
Perl, and its massive CPAN (see \url{http://metacpan.org} for more) have many options for database interaction.
%TODO link could be a footnote
Mojolicious is agnostic to the Model layer, though some suggestions for useful modules are \cpan{DBIx::Connector} for connection management or \cpan{DBIx::Class} as an ORM.
%TODO ext link about ORM
Also worthy of a mention is the Mojolicious spin-off project \cpan{Mango}, a non-blocking interface to MongoDB.

Mojolicious applications typically are a well structured set of files, defining Perl classes, templates, tests and other supporting files.
These classes represent your application and its controllers, in addition to any model logic that you might include.
%TODO links to both here
While using this full structure is highly encouraged for production, for fast prototyping, and especially for presenting examples, this structure becomes cumbersome.

Luckily, Mojolicious comes with tiny wrapper layer called Mojolicious::Lite.
This module provides a functional dsl wrapping routing methods and helpers --- which will be discussed in due course --- so that simple applications can be defined in a single script.
With this sugar, a ``hello world'' app looks something like this:
\begin{mojolite}
#!/usr/bin/env perl
use Mojolicious::Lite;

any '/' => sub { shift->render( text => 'hello world' ) };

app->start;
\end{mojolite}


